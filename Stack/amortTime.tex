\documentclass{article}
\usepackage[utf8]{inputenc}
\usepackage[T1]{fontenc}
\usepackage[russian]{babel}

\begin{document}
    Обозначим за потенциал $p$, который определим следующим образом, в зависимости от $\alpha=\frac{\text { size }}{\text { capacity }}$ :
    $$
        p_1 =2 \cdot \text{ size }- \text{ capacity ,} \alpha \geq \frac{1}{2} \\
    $$
    $$
        p_2=\frac{1}{2} \cdot \text { capacity }- \text { size, } \alpha<\frac{1}{2}.\\
    $$
    
    Массив сузится при $\alpha=\frac{1}{4}$. В таком случае $t_i=$ size -1. realloc программа по очереди будет копировать каждый элемент в другой массив.
    $$
    a_i=t_i+ p_2\left(\frac{1}{2} \text { cap }, \text { size }-1\right)-p_2(\text { cap }, \text { size })=\\
    $$
    $$
    = \text { size }-1+\left(\frac{1}{4} \operatorname{cap}-(\text { size }-1)\right)-\left(\frac{1}{4} \operatorname{cap}-\text { size }\right)=0 .
    $$
    Рассмотрим теперь случай $\frac{1}{4}<\alpha<\frac{1}{2}$ :
    $$
    a_i=t_i+p_2(\operatorname{cap}, \text { size }-1)-p_2(\operatorname{cap}, \text { size })=\\$$
    $$
    = 1+\left(\frac{1}{2} \operatorname{cap}-(\text { size }-1)\right)-\left(\frac{1}{2} \operatorname{cap}-\text { size }\right)=2\\.
    $$
    
    Cлучай $\alpha \geq \frac{1}{2}$:
    \begin{itemize}
    
    \item[1)] При удалении элемента $\alpha$ остается в том же диапозоне ( $\alpha \geq \frac{1}{2}$ ):\\
    $$
    a_i=t_i+p_1(\text { cap, size }-1)-p_1(\text { cap }, \text { size })= $$
    $$ = 1+(2 \cdot(\text { size }-1)-\operatorname{cap})-(2 \cdot \text { size }- \text { cap })=-1 .
    $$
    \item[2)]При удалении элемента $\alpha$ переходит из $\left[\frac{1}{2} ; 1\right]$ в $\left[\frac{1}{4} ; \frac{1}{2}\right)$ :
    $$
    a_i=t_i+p_2(\operatorname{cap}, \text { size }-1)-p_1(\operatorname{cap}, \text { size }) = $$
    $$ = 1+\left(\frac{1}{2} \operatorname{cap}-(\operatorname{size}-1)\right)-(2 \cdot \operatorname{size}-\operatorname{cap})$$
    $$ = 2-3 \cdot \operatorname{size}+\frac{3}{2} \operatorname{cap} ; 
    2-3 \cdot \operatorname{size}+\frac{3}{2} \operatorname{cap} = $$
    $$ = 2-3 \cdot \alpha \cdot \operatorname{cap}+\frac{3}{2} \operatorname{cap}<2-\frac{3}{2} \operatorname{cap}+\frac{3}{2} \operatorname{cap}=2 .

    $$
\end{itemize}
    амортизационное время работы - $O(1)$.
    То есть, для поддержания такого времени работы мы должны сужать стэк в 2 раза, когда его размер меньше 4 раза.
\end{document}
